\documentclass[11pt]{article}
\usepackage[T1]{fontenc}
\usepackage{alltt}
\usepackage{moreverb}
\usepackage{epsfig}
\usepackage{makeidx}
\usepackage{changebar}

\setlength{\parskip}{0.3cm}
\setlength{\parindent}{0cm}

\def\inputfig#1{\input #1}
\def\inputtex#1{\input #1}

\input spec-macros.tex

\makeindex
\begin{document}

\section{The \texttt{flexichain} protocol}

In this section, we describe the \texttt{flexichain} protocol,
allowing client code to dynamically add elements to, and delete
elements from a sequence (or chain) of such elements. 

\subsection{The concept of a position}

A flexichain uses the concept of a ``position'', which has two
different meanings in different contexts.

The first meaning is the position at which an element is located in
the chain.  In this context, the position must have a value between 0
and $l-1$, where $l$ is the length of the chain.  This meaning is used
when an element is to be deleted or when an element is accessed or
replaced.

The second meaning is the position \textit{between two
elements} (or at one of the extreme ends of the chain).
In this context, the position may have a value between 0 and the
length of the chain inclusive. The position 0 means the beginning of
the chain (before the first element, if any), and the
position equal to the length means the end of the chain
(after the last element if any).  This meaning is used when an element
is to be inserted.

\subsection{Performance}

Accessing and replacing an element are constant-time operations. 

We guarantee linear average complexity of a sequence of insert and
delete operations provided that the position of two successive
operations in the sequence is bounded.

More specifically, the average complexity of an operation is
proportional to the difference between the position of the operation
and the position of the previous operation.

Here we actually consider distance modulo the length of the chain so
that the distance between the last position and the first is $1$.
In particular, the longest possible distance between to operations is
half of the number of elements in the chain.

The implementation will allocate more space than is necessary.
Whenever space runs out, we allocate a bigger chunk of memory to hold
the elements.  The new chunk size will be $lk$, were $l$ again is the
length of the chain and $k$ is a \textit{constant factor}.  We
multiply rather than add, because we want to guarantee the linear
average complexity of a sequence of insert and delete
operations. Typically, $k$ is somewhere between $1.5$ and $2$.  The
default value is $k=1.5$.  

We shrink the space whenever the length of the chain (the number of
elements) is significantly smaller than the available space.  By
default, the definition of ``significantly'' is that the ratio of
length to size must be less than $1/k^2$ in order for the space to be
shrunk.  Again, the new chunk size will be $lk$. 

Using the default value of $k$, this means that the amount of wasted
space can be as large as the length of the chain in the worst case,
but for a sequence of insert operations, the average wasted space is
only 25\%.

Applications that store a number of elements that does not vary much,
can choose a small value for the $k$ to waste less space.  Such
applications will have to resize the space relatively rarely so
performance will not be affected by small values of $k$. 

The space will not shrink below a minimum size (default $5$
elements).  The reason for this is to avoid to many resize operations
for small chains.  It is probably not reasonable to use a value below
around $5$, since the bookkeeping information takes up at least this
much space.  This value is also used for the initial size of the
chain.  Applications that will typically store a large number of
elements can choose a greater value for the minimum size.  Doing so
also improves performance since fewer resize operations will have to
be executed. 

\subsection{Protocol classes and functions}

Many names of operations in this section have a terminatin
``\texttt{*}'' which is meant to suggest a \emph{spread} version of
the operation.  Later (in the \texttt{flexicursor} section) we give
\emph{nospread} versions of the operations.


\Defprotoclass {flexichain}

The protocol class for flexichains. 

\definitarg {:initial-contents}
\definitarg {:element-type}
\definitarg {:fill-element}
\definitarg {:expand-factor}
\Definitarg {:min-size}

All instantiable subclasses of \cl{flexichain} accept these
initargs. 

The \cl{:initial-contents} initarg is a sequence (list, vector,
string) of objects to be stored in the \cl{flexichain} from the start.

The \cl{:element-type} initarg determines the type of the elements of
the \cl{flexichain} (default is \cl{t}). 

The \cl{:fill-element} initarg should be an object that is compatible
with the \cl{:element-type} initarg and will be used to fill
unoccupied space in the chain (to help the garbage
collector).  The default value for this initarg will be supplied by
the implementation according to the element-type given.  The
implementation will test \cl{nil}, \cl{0}, and \cl{\#$\backslash$a}.
If none of these values will work, the client must supply a value that
is compatible with \cl{:element-type}.

The \cl{:expand-factor} initarg is used to determine the factor by
which the available space will be multiplied whenever the space for
the chain is full.  Default value is $1.5$.

The \cl{:min-size} initarg determines the smallest space allocated to
hold elements of the chain.  Default value is $5$.  It is not
reasonable to supply values smaller than $5$. 

The instance created by make-instance will have a length which is that
of the sequence given by \cl{:initial-contents} or $0$ if no
\cl{:initial-contents} was given.

\Defclass {standard-flexichain}

The standard instantiable subclass of \cl{flexichain}.

\Defgeneric {nb-elements} {chain}

Return the number of elements in the flexichain \arg{chain}. 

\Deferror {flexi-error}

The base condition for all conditions that may be signaled by the
operations on flexichains.

\Deferror {flexi-position}

This condition will be signaled by operations that require a position
argument whenever that argument is out of range. 

\defgeneric {insert*} {chain position object}

Insert an object at \arg{position} of the flexichain.  If
\arg{position} is out of range (less than $0$ or greater than the
length of \arg{chain}), the \cl{flexi-position} condition will be
signaled.

\Defgeneric {delete*} {chain position}

Delete an element at \arg{position} of the flexichain.  If
\arg{position} is out of range (less than $0$ or greater than or equal
to the length of \arg{chain}), the \cl{flexi-position} condition will be
signaled.

\Defgeneric {delete-elements*} {chain position n}

Delete N elements at \arg{position} of the flexichain. If
$\arg{position}+\arg{N}$ is out of range (less than $0$ or greater
than or equal to the length of \arg{chain}, the
\cl{flexi-position-error} condition will be signaled, and nothing will
be deleted. \arg{n} can be negative, in which case elements will be
deleted before \arg{position}.

\Defgeneric {element*} {chain position}

Return the element at \arg{position} of the \arg{chain}.  If
\arg{position} is out of range (less than $0$ or greater than or equal
to the length of \arg{chain}), the \cl{flexi-position} condition will be
signaled.

\Defgeneric {(setf element*)} {object chain position}

Replace the element at \arg{position} of \arg{chain} by \arg{object}.
If \arg{position} is out of range (less than $0$ or greater than or
equal to the length of \arg{chain}), the \cl{flexi-position} condition
will be signaled.

\subsection{Stack and queue operations}

A \texttt{flexichain} can be used as a stack or as a queue with very good
performance.  In this section, we suggest a set of operations to
facilitate such use. 

\Defgeneric {push-start} {chain object}

Insert an object at the beginning of the chain.

\Defgeneric {push-end} {chain object}

Insert an object at the end of the chain.

\Defgeneric {pop-start} {chain}

Pop and return the element at the beginning of the \texttt{chain}

\Defgeneric {pop-end} {chain}

Pop and return the element at the end of the \texttt{chain}

\Defgeneric {rotate} {chain \optional (n 1)}

Rotate the elements of the \texttt{chain} so that the element that used
to be at position $n$ now is at position $0$.  With a negative value
of $n$ rotate the elements so that the element that used to be at
position $0$ now is at position $n$.  When the magnitude of $n$ is
greater than the length of the \texttt{chain}, the operation wraps
around so that it becomes equivalent to the same operation with a
value of $n$ modulo the length.  When the length of the \texttt{chain}
is less than $2$, this function does nothing. 

\section{Implementation of the \texttt{flexichain} protocol}

\subsection{Representation}

We keep elements in a vector treated as a circular gap buffer with two
sentinel elements, one before the first element of the chain (with a
position of $-1$), and one after the last element of the chain (with a
position equal to the length of the chain).  We use the word
\textit{position} to refer to the abstract position of an element in a
flexichain, and the word \textit{index} when we talk about indexes of
the gap buffer used in the implementation.  We say that an index $i$
is \emph{valid} if $0 \le i < l$ where $l$ is the size of the vector
(the vector is never of size $0$, so it is always the case that $l >
0$).

We use the term \emph{extended element} to mean a user element or a
sentinel. 

We say that the vector is \textit{full} when it contains as many
extended elements as its length (i.e., the gap has a size of $0$), and
\textit{empty} when it contains no user elements (and thus only the
sentinels) (i.e., the gap is the size as the vector minus
2).

There are three different possible configurations of the gap with
respect to the data.  Figure \ref{fig-both-contiguous} shows the case
where both the gap and the data are contiguous.  Figure
\ref{fig-data-not-contiguous} shows the case where the data is not
contiguous.  Finally, figure \ref{fig-gap-not-contiguous} shows the
case where the gap is not contiguous. 

\begin{figure}
\begin{center}
\inputfig{gap1.pstex_t}
\end{center}
\caption{\label{fig-both-contiguous} Gap and data are both contiguous}
\end{figure}

\begin{figure}
\begin{center}
\inputfig{gap2.pstex_t}
\end{center}
\caption{\label{fig-data-not-contiguous} Data is not contiguous}
\end{figure}

\begin{figure}
\begin{center}
\inputfig{gap3.pstex_t}
\end{center}
\caption{\label{fig-gap-not-contiguous} Gap is not contiguous}
\end{figure}

The implementation of a flexichain allows for the first element (i.e.,
the first sentinel with a position of $-1$) to be located at any valid
index of the vector.  For that reason, we need an index (called
\texttt{data-start}) which always indicates the index of the first
sentinel i.e.. \texttt{data-start} is always a valid index

A \textit{positional index} is an index in the vector that corresponds
to a position in the flexichain, and so is an index either of a user
object or the index of the last sentinel. 

We introduce two different indexes (always valid as well),
\texttt{gap-start} and \texttt{gap-end}.  The \texttt{gap-start} index
is the first index of the gap.  When the vector is not full, the
\texttt{gap-start} is always an index containing no extended element,
such that the previous index does contain an extended element. The
\texttt{gap-end} index is the first index beyond the gap and is always
the index of an extended element.  When the vector is not full, the
previous index does not contain an extended element.  Notice that for
certain configurations \texttt{gap-start} is smaller than
\texttt{gap-end} and for certain other configurations, the reverse is
true.

When the vector is full, \texttt{gap-start} and \texttt{gap-end} are
always equal.

\subsection{Computing and index from a position}

Step one in inserting or deleting an element is to determine an index
corresponding to the position.  Here is how it is done: we compute a
value $s$ which is equal to \texttt{gap-start} if \texttt{gap-start}
is greater than \texttt{data-start}.  Otherwise $s$ is equal to
\texttt{gap-start} plus the size of the vector.  The position is added
to \texttt{data-start}, giving the value $i$.  If $i$ is greater than
or equal to $s$, the size of the gap is added to $i$.  Finally, if $i$
is greater than or equal to the length of the vector, the length of
the vector is subtracted from $i$ (prove that the result is always a
positional index).  Call this final value of $i$ the \textit{hot
spot}.

\subsection{Moving the gap to the right place}

After determining the index from a position, we need to determine
whether the gap is in the right place.  This is the case if and only
if the hot spot is equal to \texttt{gap-end}.  If that is not the
case, we need to move the gap.

There is a case when it is particularly simple to move the gap, namely
when the vector is full.  In that case, we can just assign both
\texttt{gap-start} and \texttt{gap-end} to the value of the hot spot. 

There are two ways of getting \texttt{gap-end} to be equal to the hot
spot either move everything to the left of the hot spot even further
left, or everything to the right of the hot spot (including the hot
spot itself) even further right. We always do the one that requires
the fewest elements to be moved.  One solution will require fewer than
half the elements to be moved and the other one at least half.  

Moving to the right will require that a number of elements equal to
the difference between \texttt{gap-start} and the hot spot to be
moved, provided that \texttt{gap-start} is greater than the hot spot.
If \texttt{gap-start} is smaller than the hot spot, it is that
difference plus the size of the vector. We check whether that value is
smaller than half of \texttt{nb-elements}. 

Moving the elements requires one, two, or three calls to \cl{replace}.

\subsubsection{Moving elements to the left}

Let us first consider the case of moving elements to the left. 

Case 1: If the entire contiguous gap is to the left of the hot spot
(as in the upper half of figure \ref{fig-both-contiguous} or as in
figure \ref{fig-data-not-contiguous} with the hot spot to the right of
the gap), a single call is required.

Case 2:A single call is also required if the highest valid index is
inside the gap (as in the lower part of figure
\ref{fig-both-contiguous} and in figure \ref{fig-gap-not-contiguous})
provided that the number of elements to be moved is no greater than
the part of the gap that is flush right in the vector.

Case 3: Two calls are needed if the highest valid index is inside the
gap (as in the lower part of figure \ref{fig-both-contiguous} and in
figure \ref{fig-gap-not-contiguous}), but the number of elements to be
moved is greater than the part of the gap that is flush right in the
vector.  The first call will fill the part of the gap that is flush
right in the vector, giving the situation of the upper half of figure
\ref{fig-both-contiguous}.  The second call will be as in case 1
above. 

Case 4: Two calls are also needed if the data is not contiguous (as in
figure \ref{fig-data-not-contiguous}) and the entire contiguous gap is
to the right of the hot spot, but the number of elements to the left
of the hot spot (i.e., the index of the hot spot before the move) is
no greater than the size of the gap.  The first call will move
everything to the right of the gap so that the gap will be flush right
as in case 2 above.  The second call will move the remaining
elements.  

Case 5: Three calls are needed  if the data is not contiguous (as in
figure \ref{fig-data-not-contiguous}) and the entire contiguous gap is
to the right of the hot spot, but the number of elements to the left
of the hot spot (i.e., the index of the hot spot before the move) is
greater than the size of the gap.  The first call will move the gap
flush right, creating the situation of case 3 above (which then
requires another two calls).

\subsubsection{Moving elements to the right}

Let us now consider moving elements to the right.

Case 1: If the entire contiguous gap is to the right of the hot spot
(as in the lower half of figure \ref{fig-both-contiguous} or as in
figure \ref{fig-data-not-contiguous} with the hot spot to the left of
the gap), a single call is required.

Case 2:A single call is also required if the index 0 is
inside the gap (as in the higher part of figure
\ref{fig-both-contiguous} and in figure \ref{fig-gap-not-contiguous})
provided that the number of elements to be moved is no greater than
the part of the gap that is flush left in the vector.

Case 3: Two calls are needed if index 0 is inside the gap (as in the
lower part of figure \ref{fig-both-contiguous} and in figure
\ref{fig-gap-not-contiguous}), but the number of elements to be moved
is greater than the part of the gap that is flush left in the vector.
The first call will fill the part of the gap that is flush left in
the vector, giving the situation of the lower half of figure
\ref{fig-both-contiguous}.  The second call will be as in case 1
above.

Case 4: Two calls are also needed if the data is not contiguous (as in
figure \ref{fig-data-not-contiguous}) and the entire contiguous gap is
to the left of the hot spot, but the number of elements to the right
of the hot spot (i.e., the index of the hot spot before the move) is
no greater than the size of the gap.  The first call will move
everything to the left of the gap so that the gap will be flush left
as in case 2 above.  The second call will move the remaining
elements.  

Case 5: Three calls are needed if the data is not contiguous (as in
figure \ref{fig-data-not-contiguous}) and the entire contiguous gap is
to the left of the hot spot, but the number of elements to the right
of the hot spot is greater than the size of the gap.  The first call
will move the gap flush left, creating the situation of case 3 above
(which then requires another two calls).

\subsection{Increasing the size of the vector}

We increase the size of the vector whenever it is full and another
element needs to be added. 

When this call is made, \texttt{gap-start} and \texttt{gap-end} have
the same value.  We must preserve the position of the gap in the new
vector.

A new vector with the size of the number of required elements
multiplied by \texttt{size-multiplier} is first allocated.

Next, we copy (using a single call to \cl{replace}) all elements
before the gap to the start of the new vector.  Then we copy (using
another single call to \cl{replace}) all elements after the gap to the
end of the new vector. The value of \texttt{gap-end} is incremented by
the difference in size of the two vectors, as is \texttt{data-start}
if it was greater than or equal to \texttt{gap-end}. 

\subsection{Decreasing the size of the vector}

Again, a new vector with the size of the number of required elements
multiplied by \texttt{size-multiplier} is first allocated.

Next, we copy (using a single call to \cl{replace}) all elements
before the gap to the start of the new vector.  Then we copy (using
another single call to \cl{replace}) all elements after the gap to the
end of the new vector. The value of \texttt{gap-end} is decremented by
the difference in size of the two vectors, as is \texttt{data-start}
if it was greater than or equal to \texttt{gap-end}. 

\subsection{Inserting an object}

The insertion operation is given a position.  The semantics of the
insertion operation require that all elements having a position
greater than or equal to the one given as argument to the insertion
operation be ``moved to the right'' i.e., that they have their
positions incremented by one.

After moving the hot spot to the right place, the value of
\texttt{gap-end} is the index corresponding to the position supplied
by the call.  It should be noted that the same index will result from
a position of $0$ and from a position equal to the current length of
the chain.

But first, we need to make sure the vector is not full.  If it is, we
call the function to increase its size. 

We place the object to be inserted at the index of \texttt{gap-start}
and then increment \texttt{gap-start}.  If this operation gives a
\texttt{gap-start} equal to the size of the vector, then it is set to
$0$.

\subsection{Deleting an element}

After moving the hot spot to the right place, we need to delete the
element at \texttt{gap-end}.  We do this by replacing it by the
\texttt{fill-element} so as to avoid holding on to it in case it is no
longer referenced.  Then we increment \texttt{gap-end}.

Finally, we check whether the size of the vector should be decreased. 

\subsection{Stack and queue operations}

The stack and queue operations are implemented very efficiently.  The
\texttt{push} and \texttt{pop} operations simply call the
corresponding \texttt{insert} and \texttt{delete} operations. 

The \texttt{rotate} operation deletes from one end of the chain and
inserts on the other. 

\section{The \texttt{flexicursor} protocol}

A \textit{cursorchain} is like a flexichain, but it also keeps around
a bunch of ``flexicursors''.

\subsection{The concept of a flexicursor} 

A flexicursor is an object that corresponds to a position between two
elements of the chain.  There are two types of flexicursors,
\emph{left-sticky} and \emph{right-sticky}.  The difference between
the two is the way they behave when an object is inserted at
corresponding position. When an object is inserted at the position
corresponding to a left-sticky flexicursor, this cursor will be
positioned \emph{before} the newly inserted object, i.e., the cursor
``sticks'' to the element on its left.  When an object is inserted at
the position corresponding to a right-sticky flexicursor, this cursor
will be positioned \emph{after} the newly inserted object, i.e., the
cursor ``sticks'' to the element on its right.

Whenever an object is inserted before the position of a
cursor, the position of the cursor will be incremented.  Conversely,
whenever an element is deleted from a position below that of a cursor,
the position of the cursor is decremented.

\subsection{Mixing \texttt{flexicursor} and \texttt{flexichain} operations}

The user can freely mix editing operations from the
\texttt{flexicursor} and the \texttt{flexichain} protocol.  When an
editing operation from the \texttt{flexichain} protocol is used on an
\texttt{cursorchain} object, the cursors of the \texttt{cursorchain}
object are updated accordingly.

\subsection{Performance}

There can be a very large number of cursors in a chain without any
negative impact on performance.  In particular, a sequence of insert
operations is not affected by the number of cursors of the chain.
For insert operations, we maintain the complexity proportional to the
distance between two consecutive positions.  

A delete operation takes time proportional to the number of
left-sticky cursors to the right of the element to delete plus the
number of right-sticky cursors to the left of it. 

The only bad case is thus a delete operation of an element with an
unbounded number of cursors sticking to it. 

\subsection{Protocol classes and functions}

\Defprotoclass {cursorchain}

This is a subclass of \cl{flexichain}.

\Defclass {standard-cursorchain}

The standard instantiable subclass of \cl{cursorchain}.

\Defprotoclass {flexicursor}

The protocol class for all flexicursors.

\Definitarg {chain}

This initarg determines the cursorchain with which the cursor is associated. 

\Defclass {standard-flexicursor}

The standard instantiable subclass of \cl{flexicursor}. 

\Defclass {left-sticky-flexicursor}

The standard instantiable class for left-sticky flexicursors.  It is a
subclass of standard-flexicursor. 

\Defclass {right-sticky-flexicursor}

The standard instantiable class for right-sticky flexicursors.  It is a
subclass of standard-flexicursor. 

\Defgeneric {chain} {cursor}

Return the underlying cursorchain of the flexicursor given
as argument. 

\Defgeneric {clone-cursor} {cursor}

Create a cursor that is initially at the same location as the one
given as argument.  

\Deferror {flexi-position-error}

This condition is signaled whenever an attempt is made to use position
outside of the range of valid positions. 

\Defgeneric {cursor-pos} {cursor}

Return the position of the cursor.

\Defgeneric {(setf cursor-pos)} {position cursor}

Set the position of the cursor.  If the new position of the cursor is
before the first position or after the last position of the chain, the
condition \cl{flexi-position-error} is signaled. 

\Defgeneric {at-beginning-p} {cursor}

Return true if the cursor is at the beginning of the chain (i.e., if
it has a position of 0).  This operation is guaranteed to be executed
in O(1) time. 

\Deferror {at-beginning}

This condition is signaled whenever an attempt is made to move a
cursor beyond the beginning of the chain. 

\Defgeneric {at-end-p} {cursor}

Return true if the cursor is at the end of the chain (i.e., if it has
a position equal to the length of the chain).  This operation is
guaranteed to be executed in O(1) time.

\Deferror {at-end}

This condition is signaled whenever an attempt is made to move a
cursor beyond the end of the chain. 

\Deferror {incompatible-object-type}

This condition is signaled whenever an attempt is made to insert an
object of an incompatible type into an chain. 

\Defgeneric {insert} {cursor object}

Insert an object at the position corresponding to that of the cursor. 
All cursors located at positions greater than the one corresponding to
the cursor given as argument, as well as left-sticky cursors (possibly
including the one given as argument)  located at the same position as
the one given as argument will have their positions incremented by
one. Other cursors are unaffected. 

If the type of the object does not match the type accepted by the
underlying chain, the \cl{incompatible-object-type} condition is
signaled.

\Defgeneric {insert-sequence} {cursor sequence}

The effect is the same as if each object of the sequence were
inserted using the \texttt{insert} generic function. 

\Defgeneric {delete<} {cursor \optional (n 1)}

Delete n elements before the cursor.

\Defgeneric {delete>} {cursor \optional (n 1)}

Delete n elements after the cursor.  ...

A sequence of insert and delete operations is guaranteed to be
efficient if the positions of successive operations are not too far
apart as measured by the shortest distance of the chain viewed as a
circular list.  Thus, the beginning and the end of the chain are
considered close.

\Defmacro {with-editing-operations} {cursor \body body}

This macro can be used to group a bunch of editing operations (insert,
delete) into a body.  The sequence remains locked for the duration of
invocation.  Other cursors of the underlying chain are updated only
after the last operation has been completed, thus making it more
efficient to use this macro than to use individual editing operations.

\Defgeneric {element<} {cursor}

Return the element immediately before the cursor.  If the cursor is
at the beginning, an at-beginning condition will be signaled. 

\Defgeneric {(setf element<)} {object cursor}

Replace the element immediately before the cursor by the object given
as argument.  If the cursor is at the beginning, an at-beginning
condition will be signaled.

\Defgeneric {element>} {cursor}

Return the element immediately after the cursor.  If the cursor is
at the end, an at-end condition will be signaled. 

\Defgeneric {(setf element>)} {object cursor}

Replace the element immediately after the cursor by the object given
as argument.  If the cursor is at the end, an at-end condition will be
signaled.

\section{Implementation of the \texttt{flexicursor} protocol}

Cursors are stored as lists of weak references so that they can be
recycled when no longer referenced by client code.  A vector that
parallels the one holding elements of the flexichain holds per-element
lists of cursors that stick to that element. 

A cursor contains its \textit{index in the vector} as opposed to its
\textit{position in the sequence}.  This method avoids most updates of
cursors at each insert and delete operation.  Most cursors need only
be updated whenever the gap moves.  For left-sticky cursors, we store
the index of $p-1$, where $p$ is the position of the cursor.  For
right-sticky cursors, we store $p$ itself. 

After a delete operation, cursors with indexes equal to the old value
of \texttt{gap-end} need to be updated.  Right-sticky cursors will be
attached to the index corresponding to the new value of
\texttt{gap-end}, whereas left-sticky cursors get attached to the
position immediately preceding \texttt{gap-start}. 

Insert operations do not affect cursors at all. 

Mixing of \texttt{flexicursor} and \texttt{flexichain} editing
operations is possible thanks to an internal protocol for moving the
gap.  The \texttt{flexicursor} code uses :before, :after, and :around
methods on the \texttt{flexichain} editing operations as well as on
the code for moving the gap to update the cursors accordingly.  This
way, a \texttt{flexicursor} editing operation translates directly to a
\texttt{flexichain} editing operation with no extra code.

\end{document}