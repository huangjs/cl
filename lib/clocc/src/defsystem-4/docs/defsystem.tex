%% -*- Mode: LaTeX -*-

% defsystem.tex --

% \title{DEFSYSTEM: A \texttt{make} facility for Common Lisp}


\documentclass[a4paper]{article}

\usepackage{latexsym}
%\usepackage{epsfig}
\usepackage{alltt}

%\input{progr-macros}

%\addtolength{\textwidth}{.75in}
%\addtolength{\textheight}{.75in}

\newcommand{\CL}{\textsc{Common Lisp}}
\newcommand{\CLOS}{\textsc{CLOS}}
\newcommand{\Java}{\textsc{Java}}

\newcommand{\DEFSYSTEM}{\texttt{DEFSYSTEM}}

\newcommand{\checkcite}[1]{{\textbf{[Missing Citation: #1]}}}
\newcommand{\checkref}[1]{{\textbf{[Missing Reference: #1]}}}

\newcommand{\missingpart}[1]{{\ }\vspace{2mm}\\
{\textbf{[Still Missing: #1]}}\\
\vspace{2mm}}

\newcommand{\marginnote}[1]{%
\marginpar{\begin{small}\begin{em}
\raggedright #1
\end{em}\end{small}}}

\newcommand{\code}[1]{\texttt{#1}}
\newcommand{\object}[1]{\texttt{\textit{#1}}} % 'object' of discourse.
\newcommand{\clobject}[1]{\texttt{\textit{#1}}} % A CL 'object'.
\newcommand{\syntaxnt}[1]{\emph{<#1>}}

\newcommand{\tm}{$^\mathsf{tm}$}


\title{
\texttt{DEFSYSTEM}: A \texttt{make} for Common Lisp\\
{\normalsize A Thoughtful Re-Implementation of an Old Idea.}\\
(Draft 1)
}

\author{Marco Antoniotti and Peter Van Eynde}

%\date{}


%\includeonly{}

\begin{document}

\maketitle

\section{Introduction}

The utilities of the \texttt{make} class \missingpart{more
intro\ldots}.


\section{Structural Form}

\missingpart{quick explanation of ``structural vs. procedural''.}

\section{Component Hierarchy}

<The system, written in \emph{portable} \CLOS{}, will rely on the
class hierarchy depicted in Figure~\ref{fig:class-hier}

\begin{figure}
\rule{\textwidth}{2pt}
\begin{alltt}
system-component
        system
                system-library
                different-system
                        acl-defsystem
                        harlequin-defsystem
                        mcl-defsystem
                        pcl-defsystem
                        symbolics-defsystem
        module
                module-library
        file
                C/C++/Objc-source
                        H-source
                        C-source
                        C++-source
                        Objc-source
                OS-object-file
                        a.out-file
                        elf-file
                        coff-file
                        exe-file
                        obj-file
                Common-Lisp-source
                        Common-Lisp-PCL-dependent-source
                Scheme-source
                Java-source
                Java-class-file
         tag
\end{alltt}
\rule{\textwidth}{2pt}
\caption{The \texttt{DEFSYSTEM} base hierarchy.}
\label{fig:class-hier}
\end{figure}

The key elements in the hierarchy and in the structure of a system
definition are the top level \object{system}, \object{module}, \object{file}.
\begin{description}
\item[{\object{system}:}]
        this component is a \emph{logical} grouping of
        related and \emph{self-contained} pieces of code, which -- once made
        ready for execution -- realize a given application.

\item[{\object{module}:}]
        this component is meaningful only within a \object{system} and
        it is meant to denote a \emph{logical} subdivision of a system
        into sub-parts which are \emph{not} self-contained.

\item[{\object{file}:}]
        this component represents a traditional file on a given
        operating system.

        The various flavors of the \object{file} represent different
        file types (mostly sources and derivations from them through
        compilation).

\item[{\object{tag}:}]
        this component is a special \emph{place holder} which can be
        used to attach special code to be executed in a given
        dependency graph.
\end{description}

The object \object{system} has some flavors which serve to capture
some pragmatic cases which can present themselves in the course of
development of a complex system.

\begin{description}
\item[{\object{system-library}:}] This is just a system which will
        offer provisions to produce a \emph{library} -- where the
        meaning of \emph{library} is quite open-ended: an
        approximation would be a \code{.so} or a \code{.dll}, but also
        a specially compiled \CL{} file or a \Java{} \code{.jar} or
        \code{.zip} file.

\item[{\object{different-defsystem}:}] This case is rather common in
        \CL{} and it serves as a place holder for the various version of
        \code{DEFSYSTEM} which float around different implementations.
\end{description}

The object \object{module} has a \object{module-library} flavor which
has the same role has the \object{system-library} object, but with all
the \emph{logical} limitations that a module has with respect to a
\object{system}.

\subsection{``Part-of'' admissible relations}

A \object{system} is (most of the times) a collection of files,
directories and other items, logically organized according to a \emph{part-of}
relationship.

Each system component may contain a set of sub-components with the
following limitations:
\begin{itemize}
\item	a \object{file} cannot contain any sub-component\marginnote{Or
	maybe it should be allowed to?} and it cannot stand alone,
	i.e. it must be contained in a \object{module} or a \object{system}.
\item	a \object{module} cannot stand alone, i.e. it must be
	contained into a \object{system} or another \object{module}.
\item	a \object{system} can be contained as a sub-component in
	either another \object{system} or \object{module}.
\item	a \object{tag} does not contain any sub component and must be
	contained in a \object{system} or a \object{module}.
\end{itemize}

\subsection{File System and Package Names Mapping Conventions}

A \object{system} defines a logical collection of items (i.e. files)
which resides on a file system provided by the underlying operating
system. Therefore a system must have an associated \emph{file system
location} in some directory. Also, it is desirable to have a \CL{}
\emph{package} associated with each \object{system}.

These characteristics are similar to those found in the \Java{}
language system\footnote{Of course, the possibility -- or at least a
convention -- to define \emph{hierarchical} packages in \CL{} would be
helpful.}.

Therefore the \DEFSYSTEM{} facility will rely on a centralized place
holder of ``locations'' where to start looking for the
\object{system}s present in the machine or network. In previous
\DEFSYSTEM{}s, this feature was provided through the use of a variable
(e.g. \code{*central-registry*}); in \Java{} this would be somewhat
equivalent to the \code{CLASSPATH} machinery.

This version of \DEFSYSTEM{} constructs names of \CL{} packages and
sub-directories starting from the name of the \object{system} and/or
\object{module}, unless explicitly overridden (we will see how).

Each \object{file} object will also be charged to produce auxiliary
file names to support various operations. E.g. a C file named
\code{foo.c} will be in charge to produce either a name for its
compiled version: \code{foo.o} or \code{foo.obj} in dependence of the
underlying C compiler and operating system conventions.

\subsection{Version Control}


It is desirable to provide \DEFSYSTEM{} with the capabilities typical
of a \emph{revision control system}. Therefore \DEFSYSTEM{} will
provide a set of commands to support these capabilities.

However, it it also desirable to use revision control systems which
are already available, especially on UN*X systems.  In particular, the
file-based \texttt{RCS} and \texttt{xdelta}, and the \emph{project}-based
\texttt{CVS} and \texttt{PRCS}. How this connection will be realized
is still to be decided.


\section{Syntax and Features}

The main syntactic support for the \DEFSYSTEM{} is given by the
\code{defsystem} macro, which constructs a ``top-level''
\object{system}. Its syntax is defined as:

\begin{alltt}
  (defsystem \syntaxnt{name} &rest \syntaxnt{options})
\end{alltt}
where
\begin{description}
\item[{\code{\syntaxnt{name}}}] can be either a \CL{} \clobject{symbol} or
	a \clobject{string}.
\item[{\code{\syntaxnt{options}}}] can either be one of the following.\\
	\begin{tabular}{p{.27\textwidth}p{.60\textwidth}}
	\code{:class}
		& the \CLOS{} class that specifies this
		  \object{system} specialized class.
		  The default is \code{system}.\\[1mm]

	\code{:source-pathname}
		& the pathname of the directory that actually contains
		  the \object{system}. If not present it defaults to
		  the ``current'' directory.
		  This can obviously be a \clobject{logical
		  pathname}.\\[1mm]

	\code{:source-extension}
		& the file name \emph{extension} to be used for the
		code sources.\\[1mm]


	\code{:compiled-pathname}, \code{:binary-pathname}
		& the pathname of the directory that actually contains
		the ``compiled'' components of the \object{system}.\\[1mm]

	\code{:compiled-extension}, \code{:binary-extension}
		& the file name \emph{extension} to be used for the
		``compiled'' components of the \object{system}.\\[1mm]


	\code{:package}
		& the name of the package this \object{system} should
		be loaded in.\\[1mm]

	\code{:components}
		& a \clobject{list} of \code{\syntaxnt{sub-component
		  specifications}} (described later).\\[1mm]

	\code{:depends-on}
		& either a \object{system designator} (i.e. either a
		\CL{} \clobject{string} or \clobject{symbol}) or a
		\clobject{list} of \object{system designator}s (this
		is a restricted form of the general one available to
		other classes of \object{sub-component}s).\\[1mm]

	\code{:before}
		& a \clobject{list} of \code{\syntaxnt{before methods
		  specifications}} (described later).\\[1mm]

	\code{:after}
		& a \clobject{list} of \code{\syntaxnt{after methods
		specifications}} (described later).\\[1mm]

	\code{:around}
		& a \clobject{list} of \code{\syntaxnt{around methods
		specifications}} (described later).\\[1mm]

	\code{:override}
		& a \clobject{list} of \code{\syntaxnt{`main' methods
		specifications}} (described later).\\[1mm]

	\end{tabular}
\end{description}

\noindent
In turn, each \code{\syntaxnt{sub-component specification}} is defined as
follows, with one exception.
\begin{alltt}
  (\syntaxnt{sub-component main class} \syntaxnt{name} &rest \syntaxnt{options})
\end{alltt}
A \CL{} \clobject{string} can stand as a \code{\syntaxnt{sub-component
specification}}.  In this case it is interpreted to be of class
\code{file} with proper defaults filled in\footnote{Where ``proper''
means ``conventionally agreed upon by he developers and by the users
who submitted relevant feedback.}.

\noindent
As for the top level \code{defsystem} we have that
\begin{description}
\item[{\code{\syntaxnt{sub-component main class}}}] is one of
	\begin{description}
	\item[\code{:system}] a system
	\item[\code{:module}] a module
	\item[\code{:file}] a file
	\item[\code{:tag}] a tag
	\end{description}

\item[\code{\syntaxnt{name}}] can be either a \CL{} \clobject{symbol} or
	a \clobject{string}.

\item[\code{\syntaxnt{options}}] are similar to the case for the
	top-level \code{defsystem}, with the exceptions described
	hereafter.
	\missingpart{exception descriptions: e.g. conventions on
	pathname expansions using supplied parameters and component
	name.}
\end{description}

\section{Implementation Outline}

The implementation of \DEFSYSTEM{} is briefly described in this
section. We start with the basic requirements and continue with the
description of various aspects of the implementation.

\subsection{Registries and Paths}

In order to provide a flexible (though maybe subtle and complex) way
to manipulate \DEFSYSTEM{}s specifications we must distinguish between
two kinds of \emph{locations} where to look in in order to find
various things.

First of all, we need to list the \emph{locations} where we can find
\DEFSYSTEM{}s specifications (conventionally files of type/extension
\code{.system} containing a \code{defsystem} specification -- plus
auxiliary code).

Secondly, we need a list of \emph{locations} where to look for the actual
\DEFSYSTEM{}s components.

We also want to allow these \emph{lists} of locations to partially
overlap.

This distinction is needed because in \CL{} we want to leave maximum
freedom to the programmer (even to hang her/himself) and because in \CL{}
there are no actual provisions about where things should actually be
located.

Once a \code{defsystem} form has been evaluated (maybe by loading a
\code{.system} file), we can store its definition and its
\emph{internal state} in a centralized
\object{registry} (or \emph{repository}, the actual name is
unimportant), indexed by the \object{system}'s \emph{normalized name}.
\marginnote{Define ``normalized name''.}.

\noindent
Therefore we define three variables
\begin{description}
\item[\code{*defsystems-locations*}] a \emph{set} of
	\clobject{pathname}s and/or \clobject{logical pathnames}
	denoting a set of \emph{directories} where to look to find
	\code{defsystem}s specifications.

\item[\code{*systems-paths*}]  a \emph{set} of
	\clobject{pathname}s and/or \clobject{logical pathnames}
	denoting a set of \emph{directories} where to look to find
	\code{defsystem}s actual components.

\item[\code{*system-specifications-registry*}] a data structure
	indexed by \object{system} \object{designator}s which contains the
	known (i.e. ``loaded'') \object{system} specifications.
\end{description}
The first two variables are \emph{visible} and settable by the
user. The third is really internal in the implementation.

\missingpart{Interface to underlying tools}

\missingpart{Topological sort}

\missingpart{Design Choices: when to build the top-sort, compatibility
issues etc.}

\section{Usage, Interface and Extendibility}

\subsection{Usage}

Example

\begin{alltt}

(use-package "MAKE")

(defsystem socket-interface
   :source-pathname "code:cl-sockets" \emph{; A logical pathname.}
   :components
   ((:module c-library
       :target-pathname "libCLCSocket"
       :binary-extension ".so" \emph{; :binary-extension and :target-extension
                               ; are synonyms.}
       :class module-library
       :language (member "C" "C++" "ObjC")
       :components
       ("socket.h" "interface.h""
        ("socket.c" :depends-on "socket.h")
        \emph{;; Strings and conses with a string as first element are
        ;; implicitelytaken to be files.}
        (:file "transport.c" :depends-on "socket.h")
        (:file "client"
           :source-extension "c"
           :depends-on ("socket.h "interface.h")
           :language "C") \emph{; This is actually derived from the extension.}
        (:file "connection.c" :depends-on ("socket.h "interface.h"))
      ))
    (:module cl-library
       :source-pathname "cl-library" \emph{; As is, it is relative
                                     ; to "code:cl-sockets".}
       :components
       ((:file "socket-macros")
        (:file "internals" :depends-on "socket-macros"))
       :depends-on c-library)))

\end{alltt}

\subsection{Interface}

\subsection{Extendibility}

Extensibility in \DEFSYSTEM{} is achieved mainly through
\code{:before}, \code{:after} and \code{:around} methods on the
specified \emph{component protocol}\marginnote{This \emph{component
protocol} should be defined before, especially w.r.t. the component
types and hierarchy.}. Each component definition may contain a set of
``method specifications''. An example may be the following:

\begin{alltt}
(:module xxx
   :components
   ((:file "defsystem.lisp"
           :before ((load (dfl-file)
                       (format t ";;; Loading ~A as a system.~"
                               dfl-file))
                    (compile (dfl-file)
                       (do-something-before-compiling dfl-file)))
   )))
\end{alltt}

\noindent
Of course there are provisions for specifying these methods
``separatedly'', by means of a standard \code{defmethod} definition.

\begin{alltt}
(defmethod load-component :before
       ((dfl-file (eql (find-component 'sys
                                       'xxx
                                       "defsystem.lisp"))))
   (format t ";;; Loading ~A as a system.~" dfl-file))
\end{alltt}

\subsubsection{Extensibility Tools}

\emph{List here all the generic functions hooks and the ``support
functions'' like \code{find-component}}.

\section{\DEFSYSTEM{} Dictionary}

\section{Conclusions}

\section{Distribution Agreement}

\end{document}

% end of file -- defsystem.tex --